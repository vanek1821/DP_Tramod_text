\chapter{Úvod}

Cílem diplomové práce je vytvořit konverzní nástroj, který umožní automatizaci přípravy dat pro využití nástroje Traffic Modeller, který byl vyvinut ve spolupráci se Západočeskou Univerzitou v Plzni. Tento nástroj umožňuje pomocí svého API jednoduše a rychle simulovat a testovat různé scénáře dopravy na specifikovaném území.

K naplnění datové struktury tohoto nástroje mohou být použita volně dostupná data z OpenStreetMap. V takovém případě je pro naplnění datové struktury nástroje TrafficModeller potřeba získat data z OpenStreetMap, poté je transformovat do příslušné datové struktury nástroje TrafficModeller a následně je do něj nahrát, což umožní s daty dále pracovat a vytvářet simulace dopravních situací. 

V úvodní části této práce jsou obecně popsána geografická data. Dále je představen proces ETL se zaměřením na tento typ data. Následně jsou popsány datové struktury vstupních dat OpenStreetMap a očekávané datové struktury dat pro nástroj Traffic Modeller. Další část se detailně věnuje popisu konverze mezi datovými strukturami, ze které vychází část popisující implementovaný nástroj vykonávající konverzní algoritmus. 
 
V závěru práce jsou popsány a zhodnoceny dosažené výsledky. Součástí diskuze jsou pak návrhy na vylepšení konverzního algoritmu, datové struktury a způsob práce s nástrojem Traffic Modeller.